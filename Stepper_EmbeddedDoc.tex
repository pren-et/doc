\documentclass[a4paper,10pt,fleqn]{article}

\usepackage{src/common/layout}
\usepackage{float}
\usepackage{longtable}          % Tabellen 
\usepackage{booktabs}           % Tabellen
% Spezialpakete
\usepackage{fp}
\usepackage{tikz}
\usepackage{xcolor}
% TikZ-Bibliotheken
\usetikzlibrary{arrows}
\usetikzlibrary{shapes}
\usetikzlibrary{decorations.pathmorphing}
\usetikzlibrary{decorations.pathreplacing}
\usetikzlibrary{decorations.shapes}
\usetikzlibrary{decorations.text}
\input{src/common/booleans}

\setboolean{EMBED}{true}

\title{Dokumentation Stepper}
\newcommand{\EtPath}{src/Stepper}
\newcommand{\BIBLIOGRAPHY}{src/common/ET-Gruppe_Source}
\newcommand{\DasAndereTeam}{T27 }
\newcommand{\BLDCTeams}{T27 und T32}
\newcommand{\BLDCcollab}{Dieses Kapitel ist eine Zusammenarbeit der Gruppen \BLDCTeams. }

\begin{document}

\maketitle
\clearpage
\tableofcontents
\clearpage

\input{src/common/Hardwarezusammenarbeit}
\newpage
\input{src/Stepper/Wirkungsweise}
\newpage
\input{src/Stepper/Ansteuerung}
\newpage
\input{src/Stepper/L6480}
\newpage
\input{src/Stepper/Ausblick}
\newpage
\setlength\bibitemsep{1.5\itemsep}
%Folgende Zeile auskommentieren, damit nur gebrauchte Literatur erscheint
\nocite{*}
\bibliography{\BIBLIOGRAPHY}
%\bibliographystyle{apacite}



\begin{flushleft}
    \renewcommand{\refname}{Literatur- und Quellenverzeichnis}
%        \{\refname}{Quellenverzeichnis}
    \bibliography{src/common/ET-Gruppe_Source.bib}{} %!!! Kein Leerzeichen nach dem , !!!!
\end{flushleft}

\end{document}
