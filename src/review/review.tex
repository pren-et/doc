\section{Fazit}
    Die teamübergreifende Zusammenarbeit in PREN-ET hat sich extrem bewährt. 
    Auf diese Weise konnten komplexere und anspruchsvollere Lösungen 
    realisiert werden. Dementsprechend war der Lerneffekt massiv grösser. Die 
    ET-Zusammenarbeit hat sich nicht nur in der Erhöhung der 
    \enquote{man-power} niedergeschlagen, sondern auch in der Vielfalt der 
    Themen und deren spezifischen Problemen respektive Lösungen. So konnte zum 
    Beispiel die Ansteuerung eines Bürstenlosen Motors von Grund auf 
    erarbeitet und realisiert werden. Für die Ansteuerung des Schrittmotors 
    wurde ein Integrierter Treiber mit aufwendigem Interface verwendet, 
    welches dank der Zusammenarbeit komplett implementiert werden konnte. Beim 
    Review von Schemas, finden von Fehlern im Code und beim Erstellen und 
    Korrigieren der Dokumentation war es ein grosser Vorteil, dass mehrere 
    Elektrotechnik-Studierende zusammenarbeiten konnten. 
    
    \noindent
    Zu Beginn war es zeitintensiv, die Gruppen, die Tools und das gemeinsame 
    Vorgehen zu definieren und umzusetzen. Sobald dies erledigt war, 
    funktionierte die Zusammenarbeit innerhalb von PREN-ET ausserordentlich 
    gut. Als ein weiterer kritischer Punkt bei teamübergreifenden Arbeiten 
    stellte sich die Dokumentation heraus. So muss eine Dokumentation 
    erarbeitet werden, welche bei jedem Team in die Teamdokumentation 
    eingegliedert werden muss. Zudem muss besonders darauf geachtet werden, 
    dass fremde Texte korrekt als solche gekennzeichnet sind. 
