\ifSTANDALONE
\section{Firmware}
\fi
\ifEMBED
\subsection{Firmware}
\fi

\subsection{Übersicht}

\subsection{Takt}
Der Referenztakt wird durch einen Quarz mit einer Frequenz von 12 MHz erzeugt. 
Dieser Takt dient als Referenztakt für die PLL des HS9S08JM60. Dazu wird er 
durch 8 geteilt um im erlaubten Bereich\footnote{Die Frequenz des 
Eingangssignals der PLL muss im Bereich 1 \ldots 2 MHZ liegen. \cite[p. 
195]{Datasheet:HCS08}} für die PLL zu liegen. In der PLL wird der Takt mit 32 
multipliziert. Dies ergibt einen CPU Takt von 48 MHz. Da der Bustakt der 
Hälfte des CPUtakts entspricht, weist dieser eine Frequenz von 24 MHz auf. 
Der Externe Takt (16 MHz) wird zusätzlich als externer Referenztakt zur 
Verfügung gestellt und vom RTC verwendet. 
(siehe auch \ref{sec:rtc} \nameref{sec:rtc})

\subsection{RTC}
\label{sec:rtc}
Um regelmässig ablaufende Aufgaben zu steuern wird ein entsprechender Takt 
benötigt. Dafür wird der RTC\footnote{\textbf{R}eal \textbf{T}ime 
\textbf{Counter}} verwendet. Dieser verwendet als Takt den externen 
Referenztakt mit einer Frequenz von 16 MHz. Dieser wird mit dem Prescaler auf 
eine Frequenz von 16 kHz geteilt. Über das Modulo Register kann eine 
Periodendauer im Bereich 62.5 $\mu$s \ldots 16 ms eingestellt werden. Es wird 
zunächst eine Periodendauer von 1 ms verwendet. In der 
ISR\footnote{\textbf{I}nterrupt \textbf{S}ervice \textbf{Routine}} wird ein 
Flag gesetzt, welches in der Hauptschlaufe abgefragt wird. 

