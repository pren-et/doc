\ifSTANDALONE
\section{Firmware}
\fi
\ifEMBED
\subsection{Firmware}
\fi

\subsection{Übersicht}

\subsection{Takt}
Der Referenztakt wird durch einen Quarz mit einer Frequenz von 
12\si{\mega\hertz} erzeugt.  Dieser Takt dient als Referenztakt für die PLL 
des HS9S08JM60. Dazu wird er durch 8 geteilt um im erlaubten 
Bereich\footnote{Die Frequenz des Eingangssignals der PLL muss im Bereich 1 
\ldots 2\si{\mega\hertz} liegen. \cite[p.  195]{Datasheet:HCS08}} für die PLL 
zu liegen. In der PLL wird der Takt mit 32 multipliziert. Dies ergibt einen 
CPU Takt von 48\si{\mega\hertz}. Da der Bustakt der Hälfte des CPUtakts 
entspricht, weist dieser eine Frequenz von 24\si{\mega\hertz} auf.  Der 
Externe Takt (16\si{\mega\hertz}) wird zusätzlich als externer Referenztakt 
zur Verfügung gestellt und vom RTC verwendet.  (siehe auch Abschnitt 
\ref{sec:rtc} \nameref{sec:rtc})

\subsection{RTC}
\label{sec:rtc}
Um regelmässig abzuarbeitende Aufgaben zu steuern, wird ein entsprechender 
Takt benötigt. Dafür wird der RTC\footnote{\textbf{R}eal \textbf{T}ime 
\textbf{Counter}} verwendet. Dieser verwendet als Takt den externen 
Referenztakt mit einer Frequenz von 16\si{\mega\hertz}. Dieser wird mit dem 
Prescaler auf eine Frequenz von 16\si{\kilo\hertz} geteilt. Über das Modulo 
Register kann eine Periodendauer im Bereich 62.5\si{\micro\second} \ldots 
16\si{\milli\second} eingestellt werden. Es wird zunächst eine Periodendauer 
von 1\si{\milli\second} verwendet. In der ISR\footnote{\textbf{I}nterrupt 
\textbf{S}ervice \textbf{Routine}} wird ein Flag gesetzt, welches in der 
Hauptschlaufe abgefragt wird. 
\begin{table}[h!]
    \begin{zebratabular}{p{0.10\textwidth}p{0.06\textwidth}p{0.25\textwidth}p{0.5\textwidth}}
    \rowcolor{gray} Register & Wert & Beschreibung & Bemerkungen \\
    RTCSC &
        \verb!0x38! &
        RTC Status and Control Register &
        ERCLK, Interrupts enabled, Prescaler = $10^3$ $\to$ T$_{\text{Count}}$ = $\frac{1}{16}$\si{\milli\second} \\
    RTCMOD &
        \verb!0x0F! &
        RTC Modulo Register &
        Modulo = 15 $\to$ T$_{\text{Interrupt}}$ = 1\si{\milli\second} \\
    \end{zebratabular}
    \caption{Registerinitialisierung RTC}
    \label{tab:rtc_init}
\end{table}

\subsection{PWM}
Mit der PWM\footnote{\textbf{P}ulse \textbf{W}idth \textbf{M}odulation} werden 
die Ausgangsstufen angesteuert. Über das Puls - Pausenverhältnis wird die 
Leistung eingestellt. Damit diese Ansteuerung jedoch nicht hörbar wird, muss 
der Motor mit einer PWM Frequenz oberhalb des hörbaren Frequenzbereichs des 
Menschen angesteuert werden. Es wird eine Frequenz von 24\si{\kilo\hertz} 
verwendet.  Für das Erzeugen der PWM wird der Timer TPM2 verwendet. 
\begin{table}[h!]
    \begin{zebratabular}{p{0.10\textwidth}p{0.06\textwidth}p{0.25\textwidth}p{0.5\textwidth}}
    \rowcolor{gray} Register & Wert & Beschreibung & Bemerkungen \\
    TPM2SC &
        \verb!0x08! &
        TPM2 Status and Control Register &
        Overflow interrupt disabled, no Center-aligned PWM, Bus clock as clock 
            source, Prescaler = 1\\
    TPM2CNT &
        \verb!0x????! &
        TPM2 Counter Register &
        No initialisation \\
    TPM2MOD &
        \verb!0x03FF! &
        TPM2 Counter Modulo Register &
        1023 $\to$ $\frac{24\si{\mega\hertz}}{1024} = 23.4375\si{\kilo\hertz}$  \\
    TPM2C0SC &
        \verb!0x28! &
        TPM2 Channel 0 Status and Control Register &
        Interrupt disabled, edge aligned PWM, High-true\\
    TPM2C0V &
        \verb!0x0033! &
        TPM2 Channel 0 Value Register &
        5\si{\percent} \\
    TPM2C1SC &
        \verb!0x24! &
        TPM2 Channel 1 Status and Control Register &
        Interrupt disabled, edge aligned PWM, Low-true\\
    TPM2C1V &
        \verb!0x03CD! &
        TPM2 Channel 1 Value Register &
        5\si{\percent} \\
    \end{zebratabular}
    \caption{Registerinitialisierung TPM2}
    \label{tab:rtc_init}
\end{table}

\subsection{Kommutierungsverzögerung / Zeitmessung}
Um den exakten Kommutierungszeitpunkt einstellen zu können, und um die Zeit 
zwischen zwei Kommutierungen zu messen wird ein weiterer Timer benötigt. Dafür 
wird TPM1 verwendet. Als Taktquelle für den Timer wird der Bustakt mit einer 
Frequenz von 24\si{\mega\hertz} verwendet. Dieser wird mit dem maximal möglichen 
Prescaler von 128 geteilt. Dies ergibt eine Frequenz von 187.5\si{\kilo\hertz} 
und eine Auflösung von 5.33\si{\micro\second}. Damit ist eine maximale 
Messdauer von 349.5\si{\milli\second} möglich. 

