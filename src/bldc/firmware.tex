\ifSTANDALONE
\section{Firmware}
\fi
\ifEMBED
\subsection{Firmware}
\fi

\subsection{Übersicht}

\subsection{Takt}
Der Referenztakt wird durch einen Quarz mit einer Frequenz von 12 MHz erzeugt. 
Dieser Takt dient als Referenztakt für die PLL des HS9S08JM60. Dazu wird er 
durch 8 geteilt um im erlaubten Bereich \footnote{Die Frequenz des 
Eingangssignals der PLL muss im Bereich 1 \ldots 2 MHZ liegen. \cite[p. 
195]{Datasheet:HCS08}} für die PLL zu liegen. In der PLL wird der Takt mit 32 
multipliziert. Dies ergibt einen CPU Takt von 48 MHz. Da der Bustakt der 
Hälfte des CPUtakts entspricht, weist dieser eine Frequenz von 24 MHz auf. 
Der Externe Takt (16 MHz) wird zusätzlich als externer Referenztakt zur 
Verfügung gestellt und vom RTC verwendet. 
(siehe auch \ref{sec:rtc} \nameref{sec:rtc})

