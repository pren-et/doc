\ifSTANDALONE
\section{Realisierung}
\fi
\ifEMBED
\subsection{Realisierung}
\fi 
    \ifSTANDALONE
    \subsection{Hardware} \label{ch:Hardware}
    \fi
    \ifEMBED
    \subsubsection{Hardware} \label{ch:Hardware}
    \fi
    Der L6480 besitzt, wie im \autoref{sec:L6480} beschrieben, eine SPI- 
    Schnittstelle. Über diese Schnittstelle soll der Steppertreiber die 
    Befehle des Freedomboards erhalten. Der Steppertreiberprint wird mit 
    Stiftleisten bestückt und kann so direkt auf das Freedomboard aufgesteckt 
    werden. Es wird so keine Kabelverbindung benötigt und die Elektronik 
    bleibt kompakt.
    \begin{figure}[h]
        \centering
        \begin{tikzpicture}
        \node[above right] (img) at (0,0) {\includegraphics[width=8cm]{\EtPath/Bilder/hardware.jpg}};
        \draw[line width=1pt]
        (1.5, 0.75) node[below=1mm] {$Freedom-Board$}
        -- (2.7, 1);
        \fill (2.7,1) circle (2pt);
        \draw[line width=1pt]
        (2, 4) node[above=2mm] {$Adapterprint$}
        -- (2.7, 2);
        \fill (2.7,2) circle (2pt);
        \end{tikzpicture}
        \caption{Freedomboard und Steppertreiberprint}
        \label{fig:Steppertreiber}
    \end{figure}
    \newline
    Der Steppertreiberprint beinhaltet grundsätzlich den Treiber für den 
    Schrittmotor, realisiert mit dem Baustein L6480 und einer externen H- 
    Brücke aus N- FET. Zusätzlich sind Anschlüsse für andere Motoren (z.B. 
    BLDC- Motor) auf dem Print integiert. Im Folgenden wird die Entwicklung 
    dieses Adapterprints vogestellt. Ab Seite \pageref{sec:Schema} ist das 
    Schema der Schaltung und ab Seite \pageref{sec:PrintDesign} die 
    Beschreibung des Printdesignes zu finden. 
    % -----------------------------------------------------------------------------------
    \ifSTANDALONE
    \subsection{Schema} \label{sec:Schema}
    \fi
    \ifEMBED
    \subsubsection{Schema} \label{sec:Schema}
    \fi
    Als Motorenkontoller wird der integrierte Schaltkreis L6480 von 
    STMicroelectronics verwendet. Dieser wird über eine SPI-Schnittstelle 
    angesteuert. Der l6480 bietet die Möglichkeit, Bewegungsprofile zu 
    konfigurieren. Die verschiedenen Betriebsarten werden im Kapitel 
    \ref{ch:Software} vorgestellt. Die H- Brücke wird extern mit N-FETs 
    realisiert. Die Beschaltung des L6480 konnte aus dem Datenblatt 
    \cite{Datasheet:L6480} entnommen werden.\\\\ 
    Das revidierte Schema ist auf Seite \pageref{Schema} abgebildet.
    \\\\
    \textbf{Speisung}\\
    Es gibt zwei Möglichkeiten, wie das IC gespeist werden kann: 
    \begin{itemize}
        \item Motorenspannung, interne Spannungsregler generieren die nötige 
            Gatespannung sowie die Logikspannung
        \item Externe Spannungsregler
    \end{itemize}
    Es wurde die erste Möglichkeit, die Speisung mit nur der Motorenspannung 
    gewählt. Als Motorenspeisung wurde ??V gewählt, da ... . Zusätzliche 
    Spannungsregler sind nicht notwendig. 
    \\\\
    \textbf{Motor supply voltage compensation}\\
    Der Motorcontoller bietet die Möglichkeit, Schwankungen der 
    Motorenspannung zu erkennen und so die Amplitude des PWM- Sinussignales am 
    Schrittmotor zu regeln. Dazu muss der Eingang ADCIN des Controllers 
    korrekt beschaltet werden. An diesem Analogeingang soll bei korrekter 
    Motorspannung 1/2 der Logikspannung anliegen. In diesem Fall bedeutet 
    dies, dass bei ??V Motorspannung 1.65V über dem Widerstand R4 anliegen 
    müssen. Dimensioniert wurde für R4 ?? und für R3 ??.
    \\\\\
    \textbf{LED Speisung}\\
    Damit sofort ersichtlich ist, ob der Adapterprint gespeist ist, wurde eine 
    LED vorgesehen. Diese wird zwischen die Speisung über einen Vorwiderstand 
    an GND angeschlossen. Somit ist alles in Ordnung, falls die LED leuchtet.
    \\\\
    \textbf{LED Fehler}\\
    Der Motorenkontroller kann so konfiguriert werden, dass verschiedene 
    Fehler angezeigt werden können. Dies beinhaltet zum Beispiel 
    Schrittverluste, Überstrom, Unterspannungserkennung oder Überhitzung des 
    Kontrollers. Dazu dient ein Open-Drain Pin $\overline{FLAG}$, welcher im 
    Fehlerfall auf GND zieht. Der Pin wurde so beschaltet, dass eine LED den 
    Fehlerfall auch optisch sichtbar macht. 
    \\\\
    \textbf{Pinbelegung}\\
    Die Kommunikation mit dem Motorenkontoller und die Ansteuerung des 
    Pneumatikventils wird mit dem Freedom- Board realisiert. 
    Die folgende Tabelle stellt die Schnittstelle zwischen dem Freedomboard 
    und den Anschlüssen auf dem Adapterprint dar. 
    \begin{table}[h!]
        \begin{zebralongtable}{p{3cm} p{2cm} p{1.5cm}}
             \rowcolor{gray}\textbf{Pin}                & \textbf{Stepper- board}   & \textbf{FRDM- KL25Z} \\           
            \textbf{IO L6480}\\ 
            %\cmidrule{1-1}
            $\overline{FLAG}$           & J2 Pin 2  & PTA13 \\ 
            $\overline{BUSY}$ / SYNC    & J1 Pin 2  & PTA1  \\ 
            $\overline{STBY / RESET}$   & J2 Pin 11 & PTA17 \\ 
            STCK                        & J2 Pin 9  & PTA16 \\ 
            %\cmidrule{1-1}
            \textbf{SPI L6480}          &           &       \\ 
            %\cmidrule{1-1}
            $\overline{CS}$             & J9 Pin 13 & PTE4  \\ 
            CK                          & J9 Pin 9  & PTE2  \\
            MOSI                        & J9 Pin 11 & PTE3  \\ 
            MISO                        & J2 Pin 20 & PTE1  \\ 
            \textbf{Pneumatik}\\ 
            Ventil                      & J2 Pin 18 & PTE0  \\ 
        \end{zebralongtable}
        \caption{Pinbelegung}
        \label{tab:Pinbelegung}
    \end{table} 
    \\
    \textbf{Zusätzliche Bestückung}\\
    Zusätzlich kann ein Bluetoothmodul bestückt werden. Weitere 
    Bestückungen...\\\\
    Es wurde mit dem Tool Altium Designer gearbeitet. Das überarbeitete Schema 
    des Steppertreiberprints ist auf der Seite \pageref{Schema} zu finden. 
    %*****************************************************
    % Schema Stepperboard                                %
    %*****************************************************
    \includepdf[page=1 , offset=0cm -3cm, angle = 90, width=\textwidth,picturecommand={\centering},pagecommand=\subsection*{Schema}\label{Schema}]{\EtPath/Bilder/Stepperboard_Schematic.pdf}
    \newpage
    \subsection*{Bestückungsdokumente}
    \begin{figure}[h!]
        \centering
        \begin{minipage}[hbt]{6cm}
            \centering
            \includegraphics[width=6cm]{\EtPath/Bilder/DC_Stepper0.jpg}
            \caption{Top Layer}
            \label{fig:Top Layer}
        \end{minipage}
        \hspace{1.5cm}
        \begin{minipage}[hbt]{6cm}
            \centering
            \includegraphics[width=6cm]{\EtPath/Bilder/DC_Stepper1.jpg}
            \caption{Bottom Layer}
            \label{fig:Bottom Layer}
        \end{minipage}
        \\[4ex]
        \begin{minipage}[hbt]{6cm}
            \centering
            \includegraphics[width=6cm]{\EtPath/Bilder/DC_Stepper2.jpg}
            \caption{Werte}
            \label{fig:Werte}
        \end{minipage}
        \hspace{1.5cm}
        \begin{minipage}[hbt]{6cm}
            \centering
            \includegraphics[width=6cm]{\EtPath/Bilder/DC_Stepper3.jpg}
            \caption{Bezeichnungen}
            \label{fig:Bezeichnungen}
        \end{minipage}
    \end{figure}
    % -----------------------------------------------------------------------------------
    \ifSTANDALONE
    \subsection{Print Design} \label{sec:PrintDesign}   
    \fi
    \ifEMBED
    \subsubsection{Print Design} \label{sec:PrintDesigna}
    \fi
    % TODO Betty, Korrektur durch Clirim        
    Der Adapterprint soll auf das Freedom-Board aufgesteckt werden und 
    möglichst klein sein. Der Print hat eine Grösse von 60mm x 70mm. Darauf 
    befinden sich die Anschlüsse für die Speisung, für den Motor, für weiter 
    Motoren und für einen End- oder Notschalter. Dazu wurden stabile 
    Leiterplattenanschlüsse gewählt, welche auch die hohen Phasenströme des 
    Motors aushalten und zudem eine genug hohe Spannungsfestigkeit besitzen, 
    da die Motorenspannung bis 85V gewählt werden kann. Die eingesetzten 
    Leiterplattenanschlüsse sind für eine Spannung bis 300V und einem Strom 
    bis 8A geeignet. Dies ist ausreichend für diese Anwendung. Alle Bauteile 
    sind SMD, ausser den Leiterplattenanschlüssen und den Kondensatoren.
    \newpage
    Grundsätzlich wurden folgende Regeln beim Design eingehalten: 
    \begin{itemize}
        \item Leiterbahnbreite 20mil
        \item Abstände zwischen Leiterbahnen und Polygonen 20mil
        \item keine rechten Winkel
        \item möglichst wenig Auskreuzungen auf dem Bottomlayer
        \item genügend grosse Pads 
        \item Verbindungen möglichst kurz halten
    \end{itemize}
    Da die Pins des L6480 näher als 20mil zueinander sind, konnte dort die 
    Regel für die Leiterbahnbreite und für den Abstand nicht eingehalten 
    werden. Deshalb wurde für dieses Bauteil eigene Regeln definiert. Die 
    Leiterbahnen sind um den L6480 nur 8mil breit, werden jedoch sofort auf 
    20mil verdickt. Die Motorenphasen wurden nicht mit Leiterbahnen an die 
    Anschlüsse geführt, sondern mit breiten Polygons verbunden. Die 
    Auskreuzungen auf dem Bottomlayer sollten vermieden werden, da einerseits 
    der erste Prototyp ohne Durchkontaktierung hergestellt wurde, andererseits 
    da die GND-Fläche auf dem Bottomlayer möglichst nicht "verschnitten" 
    werden sollte, um so EMV-Störungen abzuschirmen.
    Nach dem layouten des ersten Prototypes folgte das Bestellen der Bauteile, 
    welche nicht im Elektroniklabor vorhanden waren. Die Stückliste ist in der 
    \autoref{Stückliste} zu finden. Der erste Prototyp wurde von Hand 
    durchkontaktiert. Nach dem Bestücken mit der halbautomatischen 
    Bestückungsmaschine und dem Löten im Ofen wurde der Prototyp in Betrieb 
    genommen. Die Inbetriebnahme zeigte, dass einige wenige Änderungen im 
    Layout nötig waren. Diese wurden in einer Überarbeiteten Version des 
    Adapterprints berücksichtigt. Der zweite Print wurde maschinell 
    durchkontaktiert. Die Dokumente zur ersten Version sind im Anhang zu 
    finden. Die Bestückungsdokumente sind auf Seite \pageref{fig:Bottom Layer} 
    abgebildet.
    \begin{figure}[h]
        \centering
        \includegraphics[width=5cm]{\EtPath/Bilder/printdesign.JPG}
        \caption{Printdesign mit Altium Designer}
        \label{fig:printdesign}
    \end{figure}
    \begin{table}
        \begin{zebralongtable}{p{2.5cm} p{3cm} p{2cm} p{4cm} p{1cm}} 
             \rowcolor{gray}\textbf{Bauteil}            & \textbf{Bezeichnung}      & \textbf{Lieferant} & \textbf{Bestellnummer}   &\textbf{Anz}\\
                    
            Motor Contoller             & L6480H                    & Mouser            & 511-L6480H                & 1\\   
            Shottkydiode                & BAS40-04-G                & Mouser            & 78-BAS40-04-E3-08         & 1\\ 
            Zenerdiode                  & BZX585-B3V3               & Mouser            & 771-BZX585-B3V3           & 1\\ 
            Freilaufdiode               & BAV103                    & Mouser            & 512-BAV103                & 1\\ 
            FET                         & Si4178DY                  & PTA16             & 781-SI4178DY-TI-GE3       & 9\\ 
            LED                         & LSQ976                    & Mouser            & 720-LSQ976-NR-1           & 2\\ 
        \end{zebralongtable}
        \caption{Stückliste (Bauteile nicht an Lager)} 
        \label{Stückliste}
    \end{table}  
    \newpage
    % -----------------------------------------------------------------------------------   
    \ifSTANDALONE
    \subsection{Software}   \label{ch:Software} 
    \fi
    \ifEMBED
    \subsubsection{Software} \label{ch:Softwarea}
    \fi
    Diese Software auf dem Freedom-Board dient als Schnittstellensoftware. Sie 
    nimmt die Befehle der zentralen Recheneinheit über USB/UART an und sendet 
    diese an den Motor Controller l6480 sowie das Pneumatikventil weiter. Das 
    Software wurde mit der Entwicklungsumgebung Kinetis Design Studio von 
    Freescale entwickelt. Diese Umgebung bietet ein Tool, welches Processor 
    Expert heisst. Dieses erlaubt es, Komponenten wie z.B. einen 
    Analog-Digital-Wandler in das Projekt zu integrieren. Die Konfiguration 
    solcher Komponenten kann mit dem Component Inspector vorgenommen werden, 
    ohne dass einzelne Register des Mikrokontrollers beschrieben werden 
    müssen. Die Komponenten stammen von der Internetseite 
    http://steinerberg.com/EmbeddedComponents/ und können dort gratis 
    gedownloadet werden.
    \\\\
    Die Software auf dem Freedom-Board beinhaltet folgende fünf Komponenten 
    mit ihren Aufgaben: 
    \begin{itemize}
        \item FreeRTOS:         Ein Open-Source-Echtbetriebszeitsystem für 
            Mikroprozessoren und Mikrokontroller.  
        \item Shell:            Die Schnittstelle zur zentralen Recheneinheit. 
            Sie vergleicht die von der zentralen Recheneinheit ankommenden 
            Befehle und ruft die entsprechenden Funktionen auf. 
        \item SynchroMaster:    SPI- Komponente für die Kommunikation mit dem 
            Schrittmorotren-IC L6480. 
        \item LED:              Anzeige auf dem Freedom-Boarad.
        \item Bit:              Ansteuerung des Pneumatikventils.
    \end{itemize}
    \textbf{Schrittmotorentreiber}\\
    Im Datenblatt des L4680 sind die verschiedenen Bitfolgen, welche einzelnen 
    Kommandos entsprechen, dokumentiert. In der \autoref{fig:command} ist ein 
    Beispieles eines solchen Kommandos aus dem Datenblatt des l6480 zu sehen, 
    welches den Befehl RUN beschreibt. In diesem Beispiel werden 4 Bytes 
    einzeln über die SPI- Schnittstelle gesendet: 
    \begin{enumerate}
        \item Bitfolge für das Kommando RUN, sowie die Richtung (DIR) der Bewegung
        \item Bit 16 bis 19 der Geschwindigkeit, mit welcher sich der Motor dreht
        \item Bit 8 bis 15 der Geschwindigkeit
        \item Bit 0 bis 7 der Geschwindigkeit
    \end{enumerate}
    \begin{figure}[h!]
        \centering
        \includegraphics[width=10cm]{\EtPath/Bilder/command_example.JPG}
        \caption{Beispiel eines Kommandos aus dem Datenblatt des l6480}
        \label{fig:command}
    \end{figure}
    Der Treiber beinhaltet die Funktionen, welche die richtigen Bitfolgen über 
    SPI an den L6480 senden. Fast alle dieser Funktionen wurden in 
    Zusammenarbeit mit dem PREN-ET Team von Daniel Winz implementiert.  
    \\\\
    \textbf{Shell}\\
    Für das Pneumatikventil und den Steppertreiber wurden Shellfunktionen 
    implementiert. Erhält die Shell ein Kommando von der zentralen 
    Recheneinheit, so wird die Eingabe ausgewertet und die entsprechende 
    Shellfunktion ausgeführt. Die von der zentralen Recheneinheit geparsten 
    Befehle sind:  
    \begin{itemize}
        \item l6480 run [f/r] [speed]
        \item l6480 reset
        \item l6480 softstop
        \item l6480 hardstop
        \item l6480 softhiz
    \end{itemize}
    für die Steuerung des Motor Controllers, sowie
    \begin{itemize}
        \item vent shoot
    \end{itemize}
    für die Steuerung des Pneumatikventils. 
    \\\\
    Der Befehl "l6480 run" lässt den Motor in Vorwärtsrichtung [f] oder 
    Rückwärtsrichtung [r] mit der in [speed] definierten Geschwindigkeit 
    drehen. "l6480 reset" Konfiguriert den l6480 und bestromt den Motor, so 
    dass dieser ein Haltemoment aufweist. Um den Motor ohne Haltemoment 
    bewegen zu können, wird der Befehl "l6480 softhiz" verwendet, welcher die 
    H-Brücke ausschaltet. 
    
